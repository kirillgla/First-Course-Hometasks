%% This is LaTeX document. I created it using https://www.papeeria.com/

\documentclass{article}

\begin{document}
\centerline{\bf Analysis of parallel algorithms}
$p$ is number of processors\\
$n$ is number of elements in array
\section{Synchronous}
Computation:\\
1. Compute squares in $n$ operations\\
2. Sum them in $n$ operations\\
3. Compute root in $O(1)$\\
\\
$T_{1} = 2n + O(1)$\\

\section{Pairwise summation algorithm}
Assuming $p \geq \frac{n}{2}$\\
Computation:\\
1. Compute squares in $O(1)$\\
2. Sum squares, which takes aproximately $\log_{2}{n}$ operations,\\
because on every step array is split into two equal halves\\
3. Compute root in $O(1)$\\
\\
$T_p = \log_2{n} + O(1)$\\
\\
$S_p = \frac{2n}{log_2{n}}$\\
\\
$E_p = \frac{2n}{p\log_2{n}} \leq \frac{2n}{\frac{n}{2}\log_2{n}} = \frac{4}{\log_2{n}}$\\
\\
$\lim_{p \rightarrow \infty}{E_p} = 0$\\
\section{Processor-aware summation}
If $p \geq \frac{n}{2}$, this method is equivalent to previous.\\
So, let's assume $p < \frac{n}{2}$\\
\\
Computation:\\
1. On every processor compute squares in section in $\frac{n}{p}$ operations\\
2. On every processor sum squares in $\frac{n}{p}$ operations\\
3. Sum results in \~{} $\log_2{p}$ operations (same logic as in previous algorithm)\\
4. Compute root in $O(1)$\\
\\
$T_p = \frac{2n}{p} + \log_2{p}$\\
\\
$S_p = \frac{2n}{\frac{2n}{p} + \log_2{p}} = \frac{2}{\frac{2}{p} + \frac{\log_2{p}}{n}}$\\
\\
$E_p = \frac{1}{2 + \frac{p}{n}\log_2{p}}$
\end{document}
